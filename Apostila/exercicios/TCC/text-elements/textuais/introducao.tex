% INTRODUÇÃO-------------------------------------------------------------------

\chapter{INTRODUÇÃO}
\label{chap:introducao}

Tradicionalmente é ampla a aceitação de que existem ferramentas como \textit{Spring} e \textit{Ruby on Rails} que oferecem soluções robustas e eficazes para o desenvolvimento de aplicativos \textit{web}. Estas ferramentas fazem jus à sua popularidade: é exageradamente descomplicado criar e oferecer manutenção à um aplicativo que as usa, além das garantias implícitas oferecidas por simplesmente estar usando um ambiente que é testado diariamente em condições reais. Mas, atualmente, é vivido um período de diáspora, onde ninguém consegue prever qual é o mais sensato próximo passo, devido ao fato de que as correntes circunstâncias exibem extrema volatibilidade, e a abordagem monolítica adotada por estas se mostra inflexível visto que quando surge a necessidade de adaptação, é o próprio mecanismo que precisa ser alterado, e não o código que faz uso deste. Tendo tais detalhes em mente, fica evidente a urgência de inovação.
  
Junto com a oportunidade de mudança, surge a chance não só de aprender com a experiência obtida durante todos estes anos em que estas ferramentas se mostraram a melhor opção, como também de analisar como os processos usados nos dias de hoje podem ser aperfeiçoados para a possibilidade da criação de mecanismos mais eficientes e que apresentam mais desempenho.

Para que seja atingido o propósito descrito neste trabalho, é necessária a análise do que são consideradas boas e más práticas realizadas não só por ferramentas mas também pelas linguagens hoje disponíveis. Uma das alternativas é investigar quais são os aspectos cujos contribuem com a criação de \textit{softwares} de qualidade, da mesma forma que outra pode ser resumida na reflexão sobre quais são as características e funcionalidades que mais tendem a introduzir falhas, para que se consiga reduzir a quantidade de ou até mesmo prevenir por completo possíveis situações de risco.

Para que se inicie este trabalho, primeiro será mencionado quais foram as tecnologias que serviram como base para a pesquisa, como por exemplo, computação paralela, que contribui com o aumento do desempenho do \textit{software} por meio do uso de recursos como processadores com múltiplos núcleos.

Assim que uma base for estabelecida, este será sucedido com o detalhamento de como é organizada uma estrutura que, ao invés de modelar dados, modela procedimentos. Este é o ponto central do trabalho, e é o tópico cujo receberá foco.

Os esforços descritos no decorrer deste trabalho nada mais são do que um passo em direção à novas formas de como podem ser representadas ideias em forma de código-fonte e quais são os meios mais eficazes pelos quais podem ser elaborados sistemas mais robustos e que apresentam melhor desempenho, sem fazer com que a função de desenvolvimento se torne uma tarefa hercúlea para o desenvolvedor.