\documentclass[brazil]{abntex2}
% \usepackage[english,brazilian]{babel} % determina a língua usada no texto
\usepackage[utf8]{inputenc} % determina a codificação usada (UTF8 Universal)
\usepackage[T1]{fontenc} % determina a codificação dos fontes usados na saída
\usepackage{lmodern} % melhora a qualidade dos fontes no PDF
\usepackage{graphicx} % permite incorporar imagens no texto
\usepackage{amsmath,amssymb} % fontes e símbolos matemáticos adicionais
\usepackage{indentfirst} % indentação em início do primeiro parágrafo
\usepackage{url} % permite colocar urls no texto
\usepackage{lipsum}  
\usepackage{geometry}

\hypersetup{
     	%pagebackref=true,
		pdftitle={\@title}, 
		pdfauthor={\@author},
    	pdfsubject={\imprimirpreambulo},
	    pdfcreator={LaTeX with abnTeX2},
		pdfkeywords={abnt}{latex}{abntex}{abntex2}{trabalho acadêmico}, 
		colorlinks=true,       		% false: boxed links; true: colored links
    	linkcolor=blue,          	% color of internal links
    	citecolor=blue,        		% color of links to bibliography
    	filecolor=magenta,      		% color of file links
		urlcolor=blue,
		bookmarksdepth=4
}

\geometry{
    a4paper,
    left=3cm,
    top=3cm,
    right=2cm,
    bottom=2cm
}

\title{Como Funciona o GPS}
\author{Marcos V. de Moura Lima}
\date{2018}

\begin{document}
\maketitle

\tableofcontents

% Arquivo para resolução do exercício ...(colocar número do exercício)

\chapter{INTRODUÇÃO}

O Sistema de Posicionamento Global, 
mais conhecido pela sigla \textbf{\huge {GPS}} %Aumentar fonte da palavra GPS para 25pts e a deixe em negrito 
(em inglês \foreignlanguage{english}{Global Positioning System}) %Colocar trecho em inglês [positioning system]
é um sistema de posicionamento por satélite que fornece a um aparelho receptor móvel a sua posição, assim como informação horária,
sob quaisquer condições atmosféricas, a qualquer momento e em qualquer lugar na Terra, desde que
o receptor se encontre no campo de visão de três satélites GPS

Todos os satélites são controlados pelas estações terrestres de
gerenciamento. Existe uma que é a \textit{master}%Colocar [master] em itálico
, localizada no Colorado (Estados Unidos), que, com o auxílio de cinco estações de gerenciamento
espalhadas pelo planeta, monitoram o desempenho total do sistema, corrigindo
as posições dos satélites e reprogramando o sistema com o padrão
necessário\cite{Barbosa2004}. Após o processamento de todos esses dados, as correções e
sinais de controle são transferidos de volta para os satélites.

Cada um dos satélites do \textbf{GPS} %Colocar [GPS] em negrito
transmite por rádio um padrão fixado,
que é recebido por um receptor na Terra (segmento do usuário), funcionando
como um cronômetro extremamente acurado. O receptor mede a diferença
entre o tempo que o padrão é recebido e o tempo que foi emitido. Essa
diferença, não mais do que um décimo de segundo, permite que o receptor
calcule a distância ao satélite emissor multiplicando-se a 
velocidade\footnote{aproximadamente 2,99792458.108 m/s – a velocidade da luz} do sinal %colocar nota de rodapé em [velocidade] texto = ( aproximadamente 2,99792458.108 m/s – a velocidade da luz )
pelo tempo que o sinal de rádio levou do satélite ao receptor.

% Colocar esse parágrafo na outra página
\newpage
Em 24 de março 2009 foi lançado o primeiro satélite GPS equipado com uma amostra de hardware funcionando em frequência l5.
Entre outras novidades, este satélite será o primeiro a emitir o sinal GPS numa frequência de 1176.45 MHz (1.2 GHz).\newline % Quebrar uma linha aqui

Em Geral isso é uma vantagem pois:
% Colocar frases abaixo em itens não numerados
\begin{itemize}
    \item Melhora a estrutura do sinal para melhor desempenho.
    \item Transmissão superior ao do L1 e L2 sinal.
\end{itemize}

% Centralize este parágrafo
\begin{center}
    Lorem ipsum dolor sit amet, consectetur adipiscing elit.
\end{center}

% Alinhe à esquerda
\begin{flushleft}
    Lorem ipsum dolor sit amet, consectetur adipiscing elit.
\end{flushleft}

% Alinhe à direita
\begin{flushright}
    Lorem ipsum dolor sit amet, consectetur adipiscing elit.
\end{flushright}

\chapter{DESENVOLVIMENTO}
\lipsum[1-1]

\section{GPS}
\lipsum[1-2]

\section{CHIP}
\lipsum[2-4]

\subsection{SATE}
\lipsum[1-2]

\section{VELOCIDADE}
\lipsum[2-4]

\chapter{CONCLUSÃO}
\lipsum[2-4]

\bibliography{bibliografia}
\bibliographystyle{abntex2-alf}

\end{document}