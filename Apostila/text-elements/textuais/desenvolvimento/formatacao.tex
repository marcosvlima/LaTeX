
\chapter{FORMATAÇÃO DO TEXTO}

Existem vários comandos em \LaTeX{} para definir vários estilos de texto como: alinhamento, negrito, itálico etc.

\section{Comandos de Estilo de Texto}

É possível definir o estilo de um texto e juntar comandos com declarações.
Os comandos produzem seu efeito somente sobre seu argumento.
Comandos e/ou declarações podem ser acumulados: \verb|\textbf{\itshape negrito e itálico}| produz \textbf{\itshape negrito e itálico}.

Os principais comandos e declarações são apresentados a seguir:

\begin{quadro}[!htb]
    \centering
    \caption{Comandos para Estilo de Texto.\label{qua:quadro-comandos-estilo-texto}}
        \begin{tabular}{|l|l|l|}
            \hline
                \multicolumn{1}{|c|}{\textbf{Definição}} & \multicolumn{1}{c|}{\textbf{Comando}} & \multicolumn{1}{c|}{\textbf{Declaração}} \\ \hline
                \textbf{Negrito}                         & \verb|\textbf{texto}|                 & \verb|{\bfseries ...}|                    \\ \hline
                \texttt{Máquina de escrever}             & \verb|\texttt{texto}|                 & \verb|{\ttfamily ...}|                     \\ \hline
                \textit{Itálico}                         & \verb|\textit{texto}|                 & \verb|{itshape ...}|                        \\ \hline
        \end{tabular}
\end{quadro}

\section{Alinhamento}

Por padrão todo texto em \LaTeX{} já está justificado.\\

Para centralizar um texto pode-se usar:

\begin{verbatim}
\begin{center}
    % Texto Centralizado
\end{center}
\end{verbatim}

Para alinhar o texto à direita:

\begin{verbatim}
\begin{flushleft}
    % Texto Alinhado à Esquerda
\end{flushleft}
\end{verbatim}

Para alinhar o texto à esquerda:

\begin{verbatim}
\begin{flushright}
    % Texto Alinhado à Direita
\end{flushright}
\end{verbatim}

\section{Espaçamento}

No LaTeX existem alguns tipos de espaçamentos, como: \\

 \begin{itemize}
    \item Espaçamento simples \verb|\space{}|
    \item Espaçamento Vertical: \verb|\vspace{espaço}|
    \item Espaçamento Horizontal: \verb|\hspace{espaço}|
\end{itemize}

